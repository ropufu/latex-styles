
\documentclass[11pt]{article}
\usepackage[margin=1.3in]{geometry}
\usepackage{amsmath}
\usepackage{hyperref}
\usepackage{multirow}
\usepackage{booktabs}
\usepackage{epsdice}

\usepackage{../maybedice}

\newcommand{\command}[1]{\text{\textbackslash}\texttt{#1}}
\newcommand{\param}[1]{\{\text{$\langle$}\texttt{#1}\text{$\rangle$}\}}
\newcommand{\optparam}[1]{[\text{$\langle$}\texttt{#1}\text{$\rangle$}]}


\begin{document}

\section{Motivation}
It all started with a pair of Sicherman dice.
If you want to typeset a face of a six-sided die with 8 dots on it, the \texttt{epsdice} package will fail.
So, I created a TiKZ-based alternative to \texttt{epsdice}, and extended it somewhat.


\section{Commands}
This package provides the following commands:
\begin{align*}
    &\command{maybedice}\optparam{scale}\param{number of dots} \\
    &\command{maybevardice}\optparam{scale}\param{number of dots} \\
    &\command{maybemaskdice}\optparam{scale}\param{top row, middle row, bottom row} \\
    &\command{maybeorigamidice}\param{face labels} \\
    &\command{maybeorigamidiestd} \\
    &\command{maybetinyorigamidice}\param{face labels} \\
    &\command{maybetinyorigamidiestd}
\end{align*}

The styling of the output can be customized through
\begin{align*}
    &\command{maybedice\,setdotstyle}\param{style} \\
    &\command{maybedice\,setnodestyle}\param{style} \\
    &\command{maybedice\,setborderstyle}\param{style}
\end{align*}
or reset to default by \command{maybedice\,resetstyles}.

% TODO: document geometry setup.

\section{Epsdice alternative}
\begin{center}
    \newcommand{\noepsdice}{\phantom{\epsdice{1}}}
    \newcommand{\epsdicecommand}[1]{\command{epsdice}\{\texttt{#1}\}}
    \newcommand{\maybedicecommand}[1]{\command{maybedice}\{\texttt{#1}\}}
    \begin{tabular}{@{} l c @{}}
        \toprule
        \texttt{maybedice} command  & output of \texttt{epsdice} / \texttt{maybedice} / \texttt{maybevardice}  \\ \midrule
        \maybedicecommand{1}        & \epsdice{1} / \maybedice{1} / \maybevardice{1} \\
        \maybedicecommand{2}        & \epsdice{2} / \maybedice{2} / \maybevardice{2} \\
        \maybedicecommand{3}        & \epsdice{3} / \maybedice{3} / \maybevardice{3} \\
        \maybedicecommand{4}        & \epsdice{4} / \maybedice{4} / \maybevardice{4} \\
        \maybedicecommand{5}        & \epsdice{5} / \maybedice{5} / \maybevardice{5} \\
        \maybedicecommand{6}        & \epsdice{6} / \maybedice{6} / \maybevardice{6} \\
        \maybedicecommand{7}        & \noepsdice\ / \maybedice{7} / \maybevardice{7} \\
        \maybedicecommand{8}        & \noepsdice\ / \maybedice{8} / \maybevardice{8} \\
        \maybedicecommand{9}        & \noepsdice\ / \maybedice{9} / \maybevardice{9} \\
        \bottomrule
    \end{tabular}
\end{center}

The size of output generated by \command{maybedice}, \command{maybevardice}, or \command{maybemaskdice} may be changed by using changing the optional scaling parameter which defaults to 1.0.

The \command{maybemaskdice} command takes in a comma-separated list of three ``decimal masks'' (not binary!) that determines where to draw dots on a 3-by-3 grid of the face.

For example, \command{maybemaskdice}\texttt{[1.4]}\{\texttt{100, 001, 001}\} produces \maybemaskdice[1.4]{100, 001, 001}.


\section{Origami}
In addition to individual dice faces, the command
\[
    \command{maybetinyorigamidice}\param{facelabels}
\]
provides an in-text origami of a die, where \texttt{facelabels} is a comma-separated list of the labels on the faces of the die.
The standard die would look like this: \maybetinyorigamidice{2, 3, 5, 4, 6, 1}.

The
\[
    \command{maybeorigamidice}\param{facelabels}
\]
command provides a ``full-sized'' origami of a die:
\[
    \maybeorigamidice{2, 3, 5, 4, 6, 1}
\]


\end{document}
